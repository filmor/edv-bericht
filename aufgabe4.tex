\section{Aufgabe 4}
\subsection{Kontrollfragen}

\begin{description}
    \item[Was ist der Unterschied zwischen den Befehlen:
        \lstinline'cd ~lehnertz' und \lstinline'cd ~/lehnertz']
        Ersteres versucht in das Verzeichnis des Benutzers \verb'lehnertz' zu
        wechseln, letzteres versucht dasselbe mit dem Verzeichnis
        \verb'lehnertz' im eigenen Benutzerverzeichnis.
    \item[Sie befinden sich im ihrem Homeverzeichnis. Was ist der Unterschied
        zwischen den Befehlen: \lstinline'rm -rf ./uebung/*' und \lstinline'rm
        -rf ./uebung/ *']
        Ersteres l\"oscht rekursiv alle Verzeichniseintr\"age unterhalb von
        \verb'./uebung', letzteres l\"oscht zus\"atzlich dazu auch noch das
        Verzeichnis selber und alle nicht-versteckten Dateien im aktuellen
        Verzeichnis.
    \item[Angenommen der Systemverwalter hat sich einen Spass erlaubt und das ls
        Kommando gel\"oscht. Wie k\"onnen Sie trotzdem eine Liste der Dateien im
        gegenw\"artigen Verzeichnis bekommen?]
        Zum Beispiel mit \lstinline'find . -mindepth 1 -maxdepth 1'.
    \item[Sie haben ein Verzeichnis \lstinline'${HOME}/uebung' und befinden sich in
        \lstinline'${HOME}'.  Was geschieht bei folgenden Befehlen: \dots]
        \begin{enumerate}
            \item Kopiert die Datei \verb'frage.txt' aus dem Verzeichnis
                \verb'uebung' im Benutzerverzeichnis von \verb'lehnertz' nach
                \verb'uebung' im eigenen Benutzerverzeichnis. Existiert ein
                Verzeichnis mit dem Namen, so wird die Datei in diesem abgelegt
                als \verb'frage.txt', abgelegt, ansonsten wird sie einfach unter
                dem Namen \verb'uebung' gespeichert.
            \item Dasselbe wie zuvor, allerdings wird hier eine Fehlermeldung
                ausgegeben, falls \verb'uebung' kein Verzeichnis ist.
            \item Kopiert alle Dateien aus dem Verzeichnis
                \verb'~lehnertz/uebung' nach \verb'uebung'. Ist letzteres kein
                Verzeichnis wird eine Fehlermeldung ausgegeben.
            \item Dasselbe wie zuvor (m\"oglicherweise mit etwas anders
                lautender Fehlermeldung, der Effekt ist derselbe).
        \end{enumerate}
    \item[Wozu dient die \lstinline'PATH' Variable? Sie geben unvorsichtigerweise
        \lstinline'PATH=' ein. Was funktioniert danach nicht mehr?]
        Nix.
\end{description}
