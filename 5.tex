\section{Auswertung einer Messreihe}

\lstinputlisting[language=Gnuplot]{5_auswertung.gp}

\subsection*{Ausgabe:}
Die Werte wurden nachträglich auf drei Stellen gerundet, da \verb'gnuplot' dazu
scheinbar nicht in der Lage ist.
\begin{lstlisting}
Beschleunigung:  0.897  0.899
Fehler dazu:     0.036  0.008
Geschwindigkeit: 2.014  1.949
Fehler dazu:     0.379  0.092
\end{lstlisting}
Die Ergebnisse sind also für die Beschleunigung
\begin{align*}
    a_1 &= \unit[(0.9 \pm 0.04)]{\frac{m}{s^2}} \\
    a_2 &= \unit[(0.899 \pm 0.008)]{\frac{m}{s^2}}
\end{align*}
und für die Geschwindigkeit
\begin{align*}
    v_1 &= \unit[(2 \pm 0.4)]{\frac{m}{s}} \\
    v_2 &= \unit[(1.95 \pm 0.1)]{\frac{m}{s}}
\end{align*}
In beiden Fällen überschneiden sich die Fehlerintervalle weit genug, die
Ergebnisse sind konsistent.

