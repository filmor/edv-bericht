\section{Spreadsheets und Diagramme}
Berechnet wurde
\begin{align*}
    T(r)
    \label{eqn:umlauf}
\end{align*}
und das Ergebnis dann doppeltlogarithmisch aufgetragen. Die verschiedenen
Geraden entsprechen unterschiedlichen Zentralgestirnen und verifizieren so das
dritte Keplersche Gesetz:
\begin{align*}
    \frac{T^2}{r^3} = \mathrm{const}
\end{align*}

\begin{figure}[h!]
  \begin{center}
    \includegraphics{grafiken/loglog.pdf}
  \end{center}
  \caption{Graph zum dritten Keplerschen Gesetz}
  \label{fig:loglog}
\end{figure}

