\section{Verarbeitung größerer Datenmengen}

\begin{enumerate}
  \item Die bereinigte Datei entsteht am einfachsten durch mehrere
    hintereinander mit Pipes verbundene inverse reguläre Ausdrücke:
    \begin{lstlisting}
cat CFHTLS.log | grep -i Groth | grep -i 'E99 Zenith' | grep -i nan > CFHTLS_clean.log
    \end{lstlisting}
  \item Auch diese Aufgabe ist leicht aus der Kommandozeile heraus zu erledigen:
    \begin{lstlisting}
for i in D{1..4}
do
    echo "$i: $(cat daten/CFHTLS.log | grep -c $i)"
done
    \end{lstlisting}
    Die Ausgabe des Codeschnipsels ist:
    \begin{verbatim}
D1: 2751
D2: 1898
D3: 2383
D4: 2720
    \end{verbatim}
    Die Zahl für D2 und D3 sind also wesentlich kleiner als für D1 und D4.

  \item
    \begin{enumerate}
      \item Die Statistik kann wiederum komplett aus der Kommandozeile erledigt
        werden, hierzu sind allerdings diesmal awk-Klimmzüge nötig:
        \lstinputlisting{6_statistik.sh}
        Es wird einfach für jede Zeile die achte Spalte nach der Trennung
        "`erster Ordnung"', dem \verb'|'-Zeichen ausgegeben und von dieser die
        erste Spalte verarbeitet indem Summe, Summe der Quadrate und Anzahl
        gespeichert und am Ende dann zu den geforderten Werten verrechnet
        werden.

        Die Ausgabe dazu:
        \begin{verbatim}
D1: stddev:  0.238729  avg:  0.837045
D2: stddev:  0.296664  avg:  0.8696
D3: stddev:  0.243518  avg:  0.831804
D4: stddev:  0.205007  avg:  0.830496        
        \end{verbatim}
      \item Für den zweiten Teil der Aufgabe brauchen wir erstmal einen
        anständigen Datensatz, den uns wiederum awk liefert:
        \begin{lstlisting}
cat CFHTLS.log | awk -F'|' '{ print $10,$8 }' > daten.txt
        \end{lstlisting}
        Diese können wir dann mit gnuplot plotten und erhalten den folgenden
        Graphen:
          \begin{center}
            \includegraphics{grafiken/cfhtls}
          \end{center}
        Man sieht bei etwa 2400 Tagen, also etwa im Juli/August des Jahres 2005
        eine Verringerung der allgemeinen Streuung um die x-Achse, dort wurde also
        wohl das Instrument ausgetauscht.

        Um das zu verifizieren werden noch Mittelwert und Standardabweichung mit
        einem ähnlichen Skript wie oben berechnet:
        \lstinputlisting{6_statistik2.sh}
        Die Ausgabe dazu:
        \begin{verbatim}
Vorher:  avg= -0.0800357  stddev= 0.0259657
Nachher: avg= -0.0604951  stddev= 0.0411328
        \end{verbatim}
    \end{enumerate}
\end{enumerate}
