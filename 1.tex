\section{Befehlsreferenzen}
\subsection{Linux}
\begin{tabular}[h!]{cl}
  ls     & Listet ein Verzeichnis auf \\
  pwd    & Gibt das Arbeitsverzeichnis aus \\
  mkdir  & Erstellt ein Verzeichnis \\
  rmdir  & Loescht ein leeres Verzeichnis \\
  cp     & Kopiert Dateien \\
  mv     & Verschiebt Dateien \\
  rm     & Löscht Dateien \\
  du     & Zeigt den Speicherplatzbedarf der Datei an \\
  find   & Macht ganz viel magisches mit dem Dateisystem \\
  grep   & Filtert Zeilen anhand von regulären Ausdrücken \\
  ssh    & Gibt Zugriff auf eine entfernte Shell über eine sichere Verbindung \\
  scp    & Kopiert über einen ssh-Kanal Dateien \\
  wget   & Lädt Dateien von http- und ftp-Servern herunter \\
  echo   & Echo, Echo \ldots \\
  cat    & Gibt Dateien aus \\
  awk    & Interpreter für AWK (bei Linux meistens gawk) \\
  less   & Pager \\
  who    & Gibt Informationen über alles was so gerade auf dem System
           herumstreunt
\end{tabular}

\subsection{Gnuplot}
\begin{tabular}[h!]{cl}
  plot           & Plottet eine Funktion \\
  set terminal   & Legt das Ausgabeterminal fest, z.B. X11 oder PDF \\
  set output     & Legt die Ausgabedatei fest (bei PDF, PNG oder ähnlichem) \\
  fit \ldots via & Fittet eine Funktion an einen Datensatz an \\
  set logscale   & Setzt eine Achse auf logarithmische Skala \\
  \ldots w errorbars & Aktiviert Fehlerbalken \\
  \ldots u 1:2   & Wählt die ersten beiden Spalten als Daten aus
\end{tabular}

