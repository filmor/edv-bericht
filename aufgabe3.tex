\section{Aufgabe 3}
\subsection{Analyse von Pipelines}
Zu betrachten ist die Pipeline:
% TODO pgf
\lstset{language=Bash}
\begin{lstlisting}[style=Bash]
cat tr33_emails.txt tr33_emails2.txt | tr '<>' '\n' | grep '@' | sort | uniq -i
\end{lstlisting}
\begin{enumerate}
    \item Im ersten Teil werden die beiden Dateien aneinandergef\"ugt und in die
        Standardausgabe ausgegeben
    \item Im zweiten Teil werden alle Gr\"o\ss{}er- und Kleinerzeichen durch
        Zeilenumbr\"uche (\lstinline'\n') ersetzt
    \item Daraus werden nun alle Zeilen herausgeholt, die ein @-Zeichen
        enthalten \ldots
    \item \ldots, die als n\"achstes sortiert werden.
    \item Zuletzt werden dann mehrfach vorkommende Zeilen, also Email-Adressen
        (unabh\"angig von Gro\ss{}- und Kleinschreibung) entfernt.
\end{enumerate}

\begin{lstlisting}[style=Bash]
find . -type d -exec du -b {} \;| sort -n -r | head -n 10
\end{lstlisting}
\begin{enumerate}
    \item Das \lstinline'find'-Programm sammelt immer eine Liste von
        Verzeichniseintr\"agen (hier aus dem aktuellen Verzeichnis) und f\"uhrt
        bei Angabe des \lstinline'exec'-Parameters auf diesen etwas aus oder gibt sie
        einfach in die Standardausgabe (explizit mit \lstinline'-print'). Der
        Parameter \lstinline'-type d' sorgt hier daf\"ur, dass die Liste nur
        Verzeichnisse (\emph{\textbf{d}irectories}) ent\"alt.
    \item Jeder einzelne dieser Verzeichniseintr\"age wird dann \lstinline'du'
        als Parameter \"ubergeben, welches uns dann die Gr\"o\ss{}e des
        Verzeichnisses und all seiner Untereintr\"age (egal ob Datei oder
        Verzeichnis) ausgibt (das \lstinline'-b' sorgt daf\"ur, dass die
        Gr\"o\ss{}e in Bytes angegeben wird, nicht die tats\"achliche
        Gr\"o\ss{}e auf dem Datentr\"ager). In dieser Form ist der Aufruf
        ziemlich sinnlos, vermutlich sollte auch der Parameter \lstinline'-s' an
        \lstinline'du' \"ubergeben werden, der nicht weiter durch den Dateibaum
        l\"auft (was ja eigentlich \lstinline'find' bereits erledigt).
    \item Die Daten werden dann numerisch (\lstinline'-n') umgekehrt
        (\lstinline'-r') sortiert. Durch ersteres werden die Zahlen auch als
        solche aufgefasst, lexikographisch w\"are n\"amlich $100 < 10$,
        letzteres sorgt daf\"ur, dass die gr\"o\ss{}ten Eintr\"age in der Liste
        oben stehen.
    \item Zuletzt nehmen wir uns von unserem Datensatz die ersten 10 Zeilen, wir
        erhalten also (mit der vorgeschlagenen \"Anderung) die 10
        gr\"o\ss{}ten Verzeichniseintr\"age im aktuellen Arbeitsverzeichnis,
        wobei trivialerweise der erste Eintrag immer \verb'.' ist.
\end{enumerate}
Sinnvoll w\"aren hier noch der Parameter \lstinline'-S' f\"ur \lstinline'du'.

\subsection{Zugriffsrechte}
Sieht man komplett von der Existenz von ACLs, Sticky- und Setuid-Bits etc.\ ab,
so gibt es unter Linux (wie unter allen anderen Unices auch) f\"ur jeden
Verzeichniseintrag die Zugriffsrechte \emph{Lesen} (\textbf Read),
\emph{Schreiben} (\textbf Write) und \emph{Ausf\"uhren} (e\textbf{X}ecute),
jeweils einzeln definiert f\"ur den Eigent\"umer (\textbf User), die Gruppe
(\textbf Group) und alle Anderen (\textbf Others).

F\"ur Dateien sind die beiden ersten Zugriffsrechte klar, das
Ausf\"uhrbarkeitsbit regelt, ob eine Shell die Datei (implizit) ausf\"uhren
kann. Das ist besonders f\"ur Bin\"ardateien relevant, da diese im Gegensatz zu
Skripten nicht anders ausge\"uhrt werden k\"onnen.

F\"ur Verzeichnisse regelt Lese- und Schreibbit, ob in das Verzeichnis selber
geschrieben oder daraus gelesen werden darf. Schreiben hei\ss{}t in diesem Fall
Dateien oder Verzeichnisse anlegen oder l\"oschen, das ver\"andern bereits
existierender Dateien wird durch deren Zugriffsrechte geregelt. Lesen ist analog
dazu den Verzeichnisinhalt auszulesen (also letztlich, ob \lstinline'ls'
funktioniert oder nicht). Das Ausf\"uhrbarkeitsbit regelt, ob auf das
Verzeichnis zugegriffen werden kann, also ob man in das Verzeichnis mit
\lstinline'cd' wechseln kann. % Vererbung

\begin{lstlisting}[style=Bash]
$ ls -ld hausaufgaben{,/*}
drwxr-xr-- 2 sauer physik131 [...] hausaufgaben
-rw-r--r-- 1 sauer physik131 [...] hausaufgaben/analysis.txt
-rw-r-xr-- 1 sauer physik131 [...] hausaufgaben/getmean
\end{lstlisting}

