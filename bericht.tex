\documentclass[12pt,a4paper]{scrartcl}
\usepackage[ngerman]{babel}
\usepackage[scale=0.8]{geometry}
\usepackage{amsmath}
\usepackage{listings}
\usepackage{units}
\usepackage{pgffor}
\usepackage{graphicx}
\usepackage[utf8]{inputenc}

\setlength{\parindent}{0pt}
\setlength{\parskip}{8pt}

\lstset
{
  language=Bash,
  showstringspaces=false,
  captionpos=b,
  tabsize=4,
  breaklines=true
}

\DeclareMathSymbol{,}{\mathpunct}{letters}{"3B}
\DeclareMathSymbol{.}{\mathord}{letters}{"3B}
\DeclareMathSymbol{\decimal}{\mathord}{letters}{"3A}

\title{Abschlussbericht zur Vorlesung \\ EDV für Physiker}
\author{Benedikt Sauer}

\begin{document}
  \maketitle
  \begin{abstract}
    Eigentlich gibt's nichts passendes für ein \lstinline'abstract' zu sagen,
    aber hier soll halt was stehen.
  \end{abstract}
  \newpage
  \tableofcontents
  \newpage

  \foreach \x in {1, ..., 6, anhang}
  {
    \input{\x.tex}
  }
  
  \newpage
  \section*{Erklärung}

  \Large{Hiermit versichere ich, dass ich den vorliegenden Bericht selbstständig
  angefertigt und nur die angegebenen Hilfsmittel benutzt habe.


  % Erklärung
\end{document}
