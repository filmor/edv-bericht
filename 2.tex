\section{Unix-Pipelines, -Zugriffsrechte und Befehle}
\subsection{Analyse von Pipelines}
Zu betrachten ist die Pipeline:
\lstset{language=Bash}
\begin{lstlisting}
cat tr33_emails.txt tr33_emails2.txt | tr '<>' '\n' | grep '@' | sort | uniq -i
\end{lstlisting}
\begin{enumerate}
    \item Im ersten Teil werden die beiden Dateien aneinandergef\"ugt und in die
        Standardausgabe ausgegeben
    \item Im zweiten Teil werden alle Gr\"o\ss{}er- und Kleinerzeichen durch
        Zeilenumbr\"uche (\lstinline'\n') ersetzt
    \item Daraus werden nun alle Zeilen herausgeholt, die ein @-Zeichen
        enthalten \ldots
    \item \ldots, die als n\"achstes sortiert werden.
    \item Zuletzt werden dann mehrfach vorkommende Zeilen, also Email-Adressen
        (unabh\"angig von Gro\ss{}- und Kleinschreibung) entfernt.
\end{enumerate}

\begin{lstlisting}
find . -type d -exec du -b {} \;| sort -n -r | head -n 10
\end{lstlisting}
\begin{enumerate}
    \item Das \lstinline'find'-Programm sammelt immer eine Liste von
        Verzeichniseintr\"agen (hier aus dem aktuellen Verzeichnis) und f\"uhrt
        bei Angabe des \lstinline'exec'-Parameters auf diesen etwas aus oder gibt sie
        einfach in die Standardausgabe (explizit mit \lstinline'-print'). Der
        Parameter \lstinline'-type d' sorgt hier daf\"ur, dass die Liste nur
        Verzeichnisse (\emph{\textbf{d}irectories}) ent\"alt.
    \item Jeder einzelne dieser Verzeichniseintr\"age wird dann \lstinline'du'
        als Parameter \"ubergeben, welches uns dann die Gr\"o\ss{}e des
        Verzeichnisses und all seiner Untereintr\"age (egal ob Datei oder
        Verzeichnis) ausgibt (das \lstinline'-b' sorgt daf\"ur, dass die
        Gr\"o\ss{}e in Bytes angegeben wird, nicht die tats\"achliche
        Gr\"o\ss{}e auf dem Datentr\"ager). In dieser Form ist der Aufruf
        ziemlich sinnlos, vermutlich sollte auch der Parameter \lstinline'-s' an
        \lstinline'du' \"ubergeben werden, der nicht weiter durch den Dateibaum
        l\"auft (was ja eigentlich \lstinline'find' bereits erledigt).
    \item Die Daten werden dann numerisch (\lstinline'-n') umgekehrt
        (\lstinline'-r') sortiert. Durch ersteres werden die Zahlen auch als
        solche aufgefasst, lexikographisch w\"are n\"amlich $100 < 10$,
        letzteres sorgt daf\"ur, dass die gr\"o\ss{}ten Eintr\"age in der Liste
        oben stehen.
    \item Zuletzt nehmen wir uns von unserem Datensatz die ersten 10 Zeilen, wir
        erhalten also (mit der vorgeschlagenen \"Anderung) die 10
        gr\"o\ss{}ten Verzeichniseintr\"age im aktuellen Arbeitsverzeichnis,
        wobei trivialerweise der erste Eintrag immer \verb'.' ist.
\end{enumerate}
Sinnvoll w\"aren hier noch der Parameter \lstinline'-S' f\"ur \lstinline'du'.

\subsection{Zugriffsrechte}
Sieht man komplett von der Existenz von ACLs, Sticky- und Setuid-Bits etc.\ ab,
so gibt es unter Linux (wie unter allen anderen Unices auch) f\"ur jeden
Verzeichniseintrag die Zugriffsrechte \emph{Lesen} (\textbf Read),
\emph{Schreiben} (\textbf Write) und \emph{Ausf\"uhren} (e\textbf{X}ecute),
jeweils einzeln definiert f\"ur den Eigent\"umer (\textbf User), die Gruppe
(\textbf Group) und alle Anderen (\textbf Others).

F\"ur Dateien sind die beiden ersten Zugriffsrechte klar, das
Ausf\"uhrbarkeitsbit regelt, ob eine Shell die Datei (implizit) ausf\"uhren
kann. Das ist besonders f\"ur Bin\"ardateien relevant, da diese im Gegensatz zu
Skripten nicht anders ausge\"uhrt werden k\"onnen.

F\"ur Verzeichnisse regelt Lese- und Schreibbit, ob in das Verzeichnis selber
geschrieben oder daraus gelesen werden darf. Schreiben hei\ss{}t in diesem Fall
Dateien oder Verzeichnisse anlegen oder l\"oschen, das ver\"andern bereits
existierender Dateien wird durch deren Zugriffsrechte geregelt. Lesen ist analog
dazu den Verzeichnisinhalt auszulesen (also letztlich, ob \lstinline'ls'
funktioniert oder nicht). Das Ausf\"uhrbarkeitsbit regelt, ob auf das
Verzeichnis zugegriffen werden kann, also ob man in das Verzeichnis mit
\lstinline'cd' wechseln kann.

Die gestellte Aufgabe ist wie folgt erledigt (ein paar Zugriffsrechte waren
allerdings nicht eindeutig in der Aufgabenstellung festgelegt):
\begin{lstlisting}
$ ls -ld hausaufgaben{,/*}
drwxr-xr-- 2 sauer physik131 [...] hausaufgaben
-rw-r--r-- 1 sauer physik131 [...] hausaufgaben/analysis.txt
-rw-r-xr-- 1 sauer physik131 [...] hausaufgaben/getmean
\end{lstlisting}

\subsection{Kontrollfragen}

\subsubsection*{Was ist der Unterschied zwischen den Befehlen: \lstinline'cd
~lehnertz' und \lstinline'cd ~/lehnertz'}
Ersteres versucht in das Verzeichnis des Benutzers \verb'lehnertz' zu wechseln,
letzteres versucht dasselbe mit dem Verzeichnis \verb'lehnertz' im eigenen
Benutzerverzeichnis.

\subsubsection*{Sie befinden sich im ihrem Homeverzeichnis. Was ist der
Unterschied zwischen den Befehlen: \lstinline'rm -rf ./uebung/*' und
\lstinline'rm -rf ./uebung/ *'}
Ersteres l\"oscht rekursiv alle Verzeichniseintr\"age unterhalb von
\verb'./uebung', letzteres l\"oscht zus\"atzlich dazu auch noch das Verzeichnis
selber und alle nicht-versteckten Dateien im aktuellen Verzeichnis.

\subsubsection*{Angenommen der Systemverwalter hat sich einen Spass erlaubt und
das ls Kommando gel\"oscht. Wie k\"onnen Sie trotzdem eine Liste der Dateien im
gegenw\"artigen Verzeichnis bekommen?}
Zum Beispiel mit \lstinline'find . -mindepth 1 -maxdepth 1'.

\subsubsection*{Sie haben ein Verzeichnis \$\{HOME\}/uebung und befinden sich in
\$\{HOME\}.  Was geschieht bei folgenden Befehlen: \dots}
\begin{enumerate}
    \item Kopiert die Datei \verb'frage.txt' aus dem Verzeichnis
        \verb'uebung' im Benutzerverzeichnis von \verb'lehnertz' nach
        \verb'uebung' im eigenen Benutzerverzeichnis. Existiert ein
        Verzeichnis mit dem Namen, so wird die Datei in diesem abgelegt
        als \verb'frage.txt', abgelegt, ansonsten wird sie einfach unter
        dem Namen \verb'uebung' gespeichert.
    \item Dasselbe wie zuvor, allerdings wird hier eine Fehlermeldung
        ausgegeben, falls \verb'uebung' kein Verzeichnis ist.
    \item Kopiert alle Dateien aus dem Verzeichnis
        \verb'~lehnertz/uebung' nach \verb'uebung'. Ist letzteres kein
        Verzeichnis wird eine Fehlermeldung ausgegeben.
    \item Dasselbe wie zuvor (m\"oglicherweise mit etwas anders
        lautender Fehlermeldung, der Effekt ist derselbe).
\end{enumerate}

\subsubsection*{Wozu dient die \lstinline'PATH' Variable? Sie geben
unvorsichtigerweise \lstinline'PATH=' ein. Was funktioniert danach nicht mehr?}
Die Umgebungsvariable \lstinline'PATH' enthält eine durch Doppelpunkte getrennte
Liste von Pfaden, die die Shell nach ausführbaren Dateien durchsucht, wenn bei
der Eingabe kein Pfad explizit angegeben wurde. Insbesondere wird auf
Unix-Systemen im Allgemeinen \emph{nicht} das aktuelle Verzeichnis durchsucht.

Die Eingabe von obigen bewirkt in den meisten Shells (insbesondere sh (der
Ur-Shell, nicht der Symlink), (t)csh, zsh und bash) genau nichts. Lediglich in
der Korn Shell wird (wie die Aufgabe vermutlich gedacht war) die Variable
überschrieben, so dass nicht mehr auf die Programme in Pfaden wie \verb'/bin'
oder ähnliches zugegriffen wird. Nur die fest in die Shell eingebauten Kommandos
(bei der bash zum Beispiel \verb'cd', bei Busybox dagegen praktisch alles, auch
\verb'ls') bleiben implizit ausführbar, also ohne Angabe des gesamten Pfades.


\subsubsection*{Geben Sie eine Kommandofolge an (Pipes!), mit der Sie eine
alphabetisch sortierte Liste aller am System angemeldeten Benutzer bekommen.
Jeder Name soll in Ihrer Liste nur einmal vorkommen.}
\begin{lstlisting}
who | awk '{ print $1 }' | sort | uniq
\end{lstlisting}

\subsubsection*{Sie wollen die drei größten Dateien in dem Verzeichnisbaum
startend von Ihrem Homeverzeichnis wissen. Geben Sie eine Befehlsfolge die dies
leistet}
\begin{lstlisting}
find ~ -type f -exec du {} \; | sort -nr | head -n3
\end{lstlisting}
Der \lstinline'find'-Aufruf liefert zunächst eine Liste aller Dateien mit ihren
Größen, diese wird dann numerisch und in umgekehrter Reihenfolge sortiert und
zuletzt davon die oberen drei Einträge ausgegeben.

\subsubsection*{Sie wollen den gesamten Verzeichnisbaum unter Ihrem
Homeverzeichnis in einer tar Datei archivieren und mit gzip packen. Die Datei
soll als Backup in den jeweils anderen CIP-Pool ubertragen werden. Geben Sie die
nötigen Befehle hierzu.}
\begin{lstlisting}
tar -czpf backup.tar.gz ~
scp backup.tar.gz cipserv1.astro.uni-bonn.de:
\end{lstlisting}
Die Parameter von \lstinline'tar' bewirken dabei die Erstellung des neuen
Archivs (\textbf create), das Komprimieren mit \lstinline'gzip' (\textbf z), das
Bewahren der Zugriffsrechte (\textbf permissions) und zuletzt das Speichern in
der Datei \lstinline'backup.tar.gz' (\textbf file).

